\documentclass[11pt]{article}
\usepackage{dsfont}
\usepackage[parfill]{parskip}
\usepackage{xcolor,colortbl}
\newcommand{\gray}{\cellcolor{gray!40!white}}
\newcommand{\red}[1]{\color{red}{#1}}
\newcommand{\orange}[1]{\color{orange}{#1}}
\newcommand{\blue}[1]{\color{blue}{#1}}
\newcommand{\bfblu}[1]{\blue{\txbf{#1}}}
\newcommand{\undbluf}[1]{\bfblu{\underline{#1}}}
\usepackage[utf8]{inputenc}
\usepackage{framed}
\newcommand{\bfund}[1]{\textbf{\underline{#1}}}
\newcommand{\secund}[1]{\section{\underline{#1}}}
\newcommand{\subsecund}[1]{\subsection{\underline{#1}}}
\newcommand{\subsubsecund}[1]{\subsubsection{\underline{#1}}}
\newcommand{\texfrac}[2]{\dfrac{\textrm{#1}}{\textrm{#2}}}
\usepackage[margin=1in]{geometry}
\newcommand{\Memscans}{Members and candidates}
\newcommand{\memcan}{member or candidate}
\newcommand{\orund}[1]{\orange{\underline{#1}}}
\newcommand{\borund}[1]{\orund{\txbf{#1}}}
\usepackage{lua-visual-debug}
\newcommand{\oranb}[1]{\orange{\txbf{#1}}}
\newcommand{\qed}{\flushright$\blacksquare$\flushleft}
\usepackage{booktabs}
\newcommand{\grbf}[1]{\cellcolor{gray!40!white}\textbf{#1}}
\newcommand{\txbf}[1]{\textbf{#1}}
\newcommand{\txit}[1]{\textit{#1}}
\newcommand{\tx}[1]{\textrm{#1}}
\usepackage{amssymb}
\usepackage{graphicx}
\newcommand{\Sum}{\displaystyle\sum}
\newcommand{\orbfit}[1]{\textit{\textbf{\orange{#1}}}}
\usepackage{tabularx,ragged2e,booktabs,caption}
\newcommand{\ex}{\textbf{\underline{ex:} }}
\usepackage{subfigure}
\usepackage{multicol,tabularx,capt-of}
\usepackage{hhline}
\usepackage{mathtools}
\usepackage[shortlabels]{enumitem}
\usepackage{tikz}
\def\checkmark{\tikz\fill[scale=0.4](0,.35) -- (.25,0) -- (1,.7) -- (.25,.15) -- cycle;}
\usepackage{actuarialangle}
\newcommand{\ang}[1]{\actuarialangle{#1}}
\usepackage{multirow}
\usepackage{array}
\usepackage{cancel}
\newcolumntype{P}[1]{>{\centering\arraybackslash}p{#1}}
\newcolumntype{M}[1]{>{\centering\arraybackslash}m{#1}}
\usepackage{amsmath}
\usepackage{setspace}
\singlespacing
\graphicspath{ {./images/} }
\begin{document}

\begin{document}

\begin{center}
{\Huge {Math 502A}}

\vspace{0.2in}

{\Large {Classwork Assignment 1}}
\end{center}

\vspace{0.1in}

\section{Exercise 1.4.6}

Suppose that one card is to be selected from a deck of 20 cards that contains 10 red cards numbered from 1 to 10 and 10 blue cards numbered from 1 to 10. Let $A$ be the event that a card with an even number is selected, let $B$ be the event that a blue card is selected, and let $C$ be the even that a card with a number less than 5 is selected. Describe the sample space $S$ and describe each of the following events in words.

\begin{enumerate}[(a)]
\item $A \cap B \cap C$



$A\cap B\cap C$ describes the event where a Blue card is selected which has a value of 2 or 4. 

\item $B \cap C^C$


$B \cap C^C$ describes a blue card with a value of 5 or more. 



\item $A \cup B \cup C$


$A \cup B \cup C$ describes an even card of any color OR a blue card of any value OR a card of values LESS than 5 of any color. 


\item $A \cap \left(B \cup C\right)$


$A \cap \left(B \cup C\right)$ describes even card which is either less than 5 OR blue. 


\item $A^C \cap B^C \cap C^C$


$A^C \cap B^C \cap C^C$ describes an odd red card with a value of 5 or more. 

\end{enumerate}

\section {Exercise 1.4.8}

A simplified model of the human blood-type system has four blood types: A, B, AB, and O. There are two antigens, anti-A and anti-B, that react with a person's blood in different ways depending on the blood type. Anti-A reacts with blood types A and AB, but not with B and O. Anti-B reacts with blood types B and AB, but not with A and O. Suppose that a person's blood is sampled and tested with the two antigens.

Let $A$ be the event that the blood reacts with anti-A, and let $B$ be the event that it reacts with anti-B. Classify the person's blood type using the events $A$, $B$, and their complements.



$A$ describes having a blood type of A or AB

$A^C$ describes having a blood type of B or O. 

$B$ describes having a blood type of B or AB. 

$B^C$ describes having a blood type of A or O


\section{Exercise 1.5.14}

Suppose that, for the person tested in the previous exercise, the probability of type O blood is 0.5, the probability of type A blood is 0.34, and the probability of type B blood is 0.12.


Let $P(O)=0.5,P(A)=0.34,P(B)=0.12$ thus $P(AB)=0.04$

\begin{enumerate}[(a)]
\item Find the probability that each of the antigens (anti-A, anti-B) will react with this person's blood.


$P(\textrm{anti-a})=P(A\cup AB)=0.38$


$P(\textrm{anti-b})=P(B\cup AB)=0.16

\item Find the probability that both antigens will react with this person's blood.


$P(\textrm{anti-a}\cap\textrm{anti-b})=P(AB)=0.04$


\end{enumerate}

\section{Exercise 1.6.4 and 1.6.5}

A school contains students in grades 1, 2, 3, 4, 5, and 6. Grades 2, 3, 4, 5, and 6 all contain the same number of students, but there are twice this number in grade 1. If a student is selected at random from a list of all the students in the school:


Let $2x$ be the number of students in grade 1. Thus the school has $7x$ students. 




\begin{enumerate}[(a)]
\item What is the probability that she will be in grade 3?

$P(\textrm{grade 3})=\dfrac x{7x}=\dfrac17

\item What is the probability that she will be in an odd-numbered grade?

$P(\textrm{odd grade})=P(\textrm{1}\cup\textrm{3}\cup\textrm{5})=\dfrac{2x+x+x}{7x}=\dfrac47$


\end{enumerate}

\end{document}
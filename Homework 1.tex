\documentclass{article}
\usepackage{enumerate}
\usepackage[margin=1in]{geometry}
\usepackage[pdftex]{graphicx}
\usepackage{hyperref}

\begin{document}

\begin{center}
{\Huge {Math 502A}}

\vspace{0.2in}

{\Large {Homework Assignment 1}}
\end{center}

\vspace{0.1in}

\section{Exercise 1.4.7}
(2.5 pts - 0.5 pts per part) Answer parts (a)-(e) as written in the textbook.

\section{Exercise 1.5.7 (Modified)}

Consider two events $A$ and $B$ in sample space $S$ with $P(A) = 0.3$ and $P(B) = 0.6$.

\begin{enumerate}[(a)]
\item (1 pt) What is the maximum value of $P(A \cap B)$? Under what conditions is that maximum value obtained?
\item (0.5 pts) Under the conditions from part (a), what is $P(A | B)$?
\item (1 pt) What is the minimum value of $P(A \cap B)$? Under what conditions is that minimum value obtained?
\item (0.5 pts) Under the conditions from part (c), what is $P(A | B)$?
\end{enumerate}

\section{Exercise 1.12.1}
(1 pt) Answer the problem as written in the textbook and justify your answer.

\section{Exercise 1.12.2}
(1 pt) Answer the problem as written in the textbook.

\section{Exercise 2.1.4 and Exercise 2.1.16}

\begin{enumerate}[(a)]
\item (1.5 pts) Draw a tree diagram to graphically represent the problem in Exercise 2.1.4. Fill in the probabilities.
\item (1 pt) Use the tree diagram to compute the probability requested in Exercise 2.1.4.
\item (1 pt) Draw a tree diagram to graphically represent the problem in Exercise 2.1.16. Fill in the probabilities.
\item (1 pt) Use the tree diagram to compute the probability requested in Exercise 2.1.16.
\end{enumerate}

\section{Exercise 2.1.9}
(2 pts - 1 pt per part) Answer parts (a) and (b) as written in the textbook. A tree diagram may be useful for organizing your probabilities, but it is not necessary to create one.

\section{Harvard Undergraduate Admissions}

In 2010-2015, 5.45\% of domestic applicants to Harvard through the regular admissions process were admitted. However, 43.93\% of domestic applicants to Harvard through a special admissions process (for athletes, legacies, etc.) were admitted.

\begin{enumerate}[(a)]
\item (1 pt) Suppose that an admission officer first picks a process (regular or special) at random, then picks an applicant at random who applied through that process. What is the probability that the randomly selected applicant is admitted?
\item (1 pt) Suppose that instead the admissions officer randomly picks an applicant from the population of all applicants. If 4.84\% of applicants applied through the special admissions process, what is the probability that the randomly selected applicant is admitted?
\item (1 pt) Let $p$ represent the probability that an applicant applies through the special admissions process. We are interested in finding the probability that a randomly selected applicant is admitted. Find a general formula for this probability in terms of $p$.
\end{enumerate} 

\section{Exercise 2.2.7}
(2 pts - 1 pt per part) Answer parts (a) and (b) as written in the textbook.


\section{Exercise 2.5.24}
(2 pts) Answer the problem as written in the textbook. A tree diagram may be useful for organizing your probabilities, but it is not necessary to create one.

\section{Testing for Breast Cancer}

A biomarker test for breast cancer is accurate in 95\% of women with breast cancer and 99\% of women without breast cancer (that is, 95\% of women with breast cancer will test positive, and 99\% of women without breast cancer will test negative). Suppose that before taking the test, a woman believes that she has a 1 in 2500 chance of having breast cancer.

\begin{enumerate}[(a)]
\item (2 pts) If the test comes back positive, what should the woman believe her probability of having breast cancer to be?
\item (2 pts) Suppose that after this test comes back positive, the woman takes a different biomarker test that is accurate in 99\% of women with breast cancer and 95\% of women without breast cancer. Assuming that the results of the two tests are independent, if both tests come back positive, what should the women now believe her probability of having breast cancer is?
\end{enumerate}


\end{document}